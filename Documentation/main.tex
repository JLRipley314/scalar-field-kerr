% ****** Start of file aipsamp.tex ******
%
%   This file is part of the AIP files in the AIP distribution for REVTeX 4.
%   Version 4.1 of REVTeX, October 2009
%
%   Copyright (c) 2009 American Institute of Physics.
%
%   See the AIP README file for restrictions and more information.
%
% TeX'ing this file requires that you have AMS-LaTeX 2.0 installed
% as well as the rest of the prerequisites for REVTeX 4.1
% 
% It also requires running BibTeX. The commands are as follows:
%
%  1)  latex  aipsamp
%  2)  bibtex aipsamp
%  3)  latex  aipsamp
%  4)  latex  aipsamp
%
% Use this file as a source of example code for your aip document.
% Use the file aiptemplate.tex as a template for your document.
\documentclass[%
 prd,
% aip,
% jmp,
% bmf,
% sd,
% rsi,
amsmath,amssymb,
preprint,%
%reprint,%
%author-year,%
%author-numerical,%
% Conference Proceedings
]{revtex4-1}

\usepackage[hidelinks]{hyperref}
\usepackage{graphicx}% Include figure files
\usepackage{dcolumn}% Align table columns on decimal point
\usepackage{bm}% bold math
%\usepackage[mathlines]{lineno}% Enable numbering of text and display math
%\linenumbers\relax % Commence numbering lines

\newcommand{\edth}{\textnormal{\dh}}
\newcommand{\thorn}{\textnormal{\th}}
\newcommand{\Edth}{\textnormal{\DH}}
\newcommand{\Thorn}{\textnormal{\TH}}

\usepackage[utf8]{inputenc}
\usepackage[T1]{fontenc}
\usepackage{mathptmx}

\begin{document}

%\preprint{AIP/123-QED}

\title{
   The dynamics of scalar fields around (nearly) extremal Kerr black holes
}
% Force line breaks with \\

\author{Justin L. Ripley}
 \email{lloydripley@gmail.com}
\affiliation{ 
DAMTP,
Centre for Mathematical Sciences,
University of Cambridge,
Wilberforce Road, Cambridge CB3 0WA, UK.
}%

\date{\today}% It is always \today, today,
             %  but any date may be explicitly specified

\begin{abstract}
   A brief description of the code, and the equations of motion.
\end{abstract}

\maketitle

%\begin{quotation}
%The ``lead paragraph'' is encapsulated with the \LaTeX\ 
%\verb+quotation+ environment and is formatted as a single paragraph before the first section heading. 
%(The \verb+quotation+ environment reverts to its usual meaning after the first sectioning command.) 
%Note that numbered references are allowed in the lead paragraph.
%
%The lead paragraph will only be found in an article being prepared for the journal \textit{Chaos}.
%\end{quotation}
%=============================================================================
\section{\label{sec:introduction}Introduction}
We use $-+++$ signature for the spacetime.
%=============================================================================
\section{\label{sec:scalar_fields}Scalar fields}
   We consider the dynamics of the scalar field
\begin{subequations}
\label{eq:action_eom}
\begin{align}
   \label{eq:action}
   S
   =
   \int d^4x\sqrt{-g}\left(
      -
      \frac{1}{2}k\left(\psi\right)g^{\mu\nu}\nabla_{\mu}\psi\nabla_{\nu}\psi
      -
      V\left(\psi\right)
   \right)
   ,\\
   \label{eq:eom}
   \implies
   \Box\psi
   +
   \frac{1}{2}\frac{k'\left(\psi\right)}{k\left(\psi\right)}
   \left(\nabla\psi\right)^2
   -
   \frac{1}{k\left(\psi\right)}V'\left(\psi\right)
   =
   0
   .
\end{align}
\end{subequations}
While it is common to consider a coupling $f\left(\psi\right)R$,
we are only considering the dynamics of a scalar field about
a fixed Kerr background, so that $R=0$.
To relate \eqref{eq:action_eom} to some well known examples, in
Brans-Dicke gravity we would have $k=\omega/\psi$ and $V=0$,
and for a massive scalar field (as is often studied in scalar
field superadiance), we'd have $k=1$, $V=m^2\psi^2/2$.
%=============================================================================
\section{\label{sec:coordinates}
   Evolution in horizon penetrating coordinates
   }
%=============================================================================
\subsection{\label{sec:ief}
   Ingoing Eddington-Finkelstein coordinates for asymptotically flat Kerr
   }
To motivate what we are looking for, we write out the scalar
field equations of motion about a Kerr spacetime, using
ingoing Eddington-Finkelstein coordinates:
\begin{align}
   ds^2
   =&
   -
   \left(1-\frac{2Mr}{\Sigma}\right)dv^2
   +
   \frac{4Mr}{\Sigma}\left(dr-a\mathrm{sin}^2\theta d\phi\right)dv
   +
   \left(1+\frac{2Mr}{\Sigma}\right)\left(
      dr^2-2a\mathrm{sin}^2\theta dr d\phi
   \right)
   \nonumber\\&
   +
   \Sigma d\theta^2
   +
   \left(
      a^2+r^2+2Ma^2\frac{r\mathrm{sin}^2\theta}{\Sigma}
   \right)
   d\phi^2
   ,
\end{align}
where
\begin{subequations}
\begin{align}
   \Sigma
   \equiv&
   r^2+a^2\mathrm{cos}^2\theta
   ,\\
   \Delta
   \equiv&
   r^2-2Mr+a^2
   .
\end{align}
\end{subequations}
Spatial infinity lies at $r\to\infty$. Compactifying spatial infinity
means that we would lose resolution of outgoing waves,
as they become blueshifted.
This isn't too big a concern for us though, as we are mainly
interested in the dynamics near the horizon.
In terms of ingoing Eddington-Finkelstein coordinates we have:
\begin{align}
   \Box\psi
   =&
   -
   \left(1+\frac{2Mr}{\Sigma}\right)\partial_v^2\psi
   +
   \frac{4Mr}{\Sigma}\partial_v\partial_r\psi
   +
   \frac{\Delta}{\Sigma}\partial_r^2\psi
   +
   \frac{2a}{\Sigma}\partial_r\partial_{\phi}\psi
   +
   \frac{1}{\Sigma}\Delta_{\mathbb{S}_2}\psi
   \nonumber\\&
   +
   \frac{2M}{\Sigma}\partial_v\psi
   +
   \frac{2\left(r-M\right)}{\Sigma}\partial_r\psi
   ,\\
   \left(\nabla\psi\right)^2
   =&
   -
   \left(1+\frac{2Mr}{\Sigma}\right)\left(\partial_v\psi\right)^2
   +
   \frac{4Mr}{\Sigma}\partial_v\psi\partial_r\psi
   +
   \frac{\Delta}{\Sigma}\left(\partial_r\psi\right)^2
   +
   \frac{2a}{\Sigma}\partial_r\psi\partial_{\phi}\psi
   \nonumber\\&
   +
   \frac{1}{\Sigma}
   \left(
      \left(\partial_{\theta}\psi\right)^2
      +
      \left(\frac{1}{\mathrm{sin}\theta}\partial_{\phi}\psi\right)^2
   \right)
   .
\end{align}
In ingoing/outgoing radial characteristic speeds of the
wave equation are
\begin{align}
   c_{\pm}
   =
   \left(1+\frac{2Mr}{\Sigma}\right)^{-1}
   \left(
   -  \frac{2Mr}{\Sigma}
      \pm
      \left[
         \frac{4M^2r^2}{\Sigma^2}
      +  \left(1+\frac{2Mr}{\Sigma}\right)\frac{\Delta}{\Sigma}
      \right]^{1/2}
   \right)
   .
\end{align}
We see that at $\Delta=0$ is the black hole horizon
(that is $r_{\pm}=m\pm\sqrt{m^2-a^2}$).
At spatial infinity ($r\to\infty$), we see that $c_{\pm}=\pm1$.
%=============================================================================
\section{\label{sec:diagnostics}
   Diagnostics
   }
To characterize energy and angular momentum flux into the black hole
we compute contractions of the stress-energy tensor
\begin{align}
   T_{\mu\nu}
   =
   k\left(\psi\right)
   \nabla_{\mu}\psi\nabla_{\nu}\psi
   -
   g_{\mu\nu}
   \left(
      \frac{1}{2}k\left(\psi\right)
      g^{\alpha\beta}\nabla_{\alpha}\psi\nabla_{\beta}\psi
      +
      V\left(\psi\right)
   \right)
   .
\end{align}
From the Killing vectors $V\equiv\partial_v$ and $\Phi\equiv\partial_{\phi}$,
we can compute the total energy and angular momentum of the scalar field on
a $v=const.$ hypersurface
\begin{subequations}
   \begin{align}
      E(v)
      \equiv&
      \int_{\Sigma_v}n^{\mu}V^{\nu}T_{\mu\nu}
      ,\\
      L(v)
      \equiv&
      \int_{\Sigma_v}n^{\mu}\Phi^{\nu}T_{\mu\nu}
      .
   \end{align}
\end{subequations}
On the horizon we consider the ``power-spectrum'' of a given scalar quantity
at a fixed radial distance $r$:
\begin{align}
   \mathcal{P}_l\left(v,r\right)
   \equiv
   \sum_{m=-l}^l \left|f_{l,m}\left(v,r\right)\right|^2
   .
\end{align}
We will be most interested in the power spectrum of the local energy density
on the black hole horizon, 
\begin{align}
   \rho
   \equiv
   n^{\mu}n^{\nu}T_{\mu\nu}
   ,
\end{align}
where $n_{\mu}dx^{\mu}=\left|g^{vv}\right|^{-1/2}dv$.
In ingoing Eddington-Finkelstein coordinates we have
\begin{align}
   \rho
   =&
   k\left(\psi\right)
   \frac{1}{2}
   \left[
      \left(n^{\alpha}\nabla_{\alpha}\psi\right)^2
      +
      h^{\alpha\beta}\nabla_{\alpha}\psi\nabla_{\beta}\psi
   \right]
   +
   V\left(\psi\right)
   ,\nonumber\\
   =&
   k\left(\psi\right)
   \frac{1}{2}\Bigg[
      \left(1+\frac{2Mr}{\Sigma}\right)
      \left(\partial_v\psi\right)^2
      +
      \frac{4Mr}{\Sigma}
      \left(\partial_v\psi\right)\left(\partial_r\psi\right)
      \nonumber\\&
      +
      \left(
         \frac{r^2+2Mr+a^2}{\Sigma}
         -
         \frac{4Mr}{2Mr+\Sigma}
      \right)
      \left(\partial_r\psi\right)^2
      \nonumber\\&
      +
      \frac{2a}{\Sigma}
      \left(\partial_r\psi\right)
      \left(\partial_{\phi}\psi\right)
      +
      \frac{1}{\Sigma}
      \left(
         \left(\partial_{\theta}\psi\right)^2
         +
         \frac{1}{\mathrm{sin}^2\theta}
         \left(\partial_{\phi}\psi\right)^2
      \right)
   \Bigg]
   +
   V\left(\psi\right)
   .
\end{align}
%=============================================================================
\appendix
%=============================================================================
\section{\label{sec:code_description}
   Code description
   }
   The code we use, which solves the equation \eqref{eq:eom},
   can be accessed at 
   \url{https://github.com/JLRipley314/scalar-field-kerr}.
   It makes use of pseudospectral methods: we use a 
   basis of Chebyshev polynomials to compute radial derivatives,
   and a basis of spherical harmonics to compute angular derivatives 
   (see Sec.~\ref{sec:properties_spherical_harmonics} for a review).
   At every time step we apply a low-pass filter in spectral space of the form
   \begin{align}
      \phi_{n,l,m}
      \to
      \mathrm{exp}\left[-f\frac{n^{p_n}}{n_{max}}\right]
      \mathrm{exp}\left[-f\frac{l^{p_l}}{l_{max}}\right]
      \mathrm{exp}\left[-f\frac{m^{p_m}}{m_{max}}\right]
      \phi_{n,l,m}
      .
   \end{align}
   After some experimentation, we found $p_i\gtrsim12$ worked well.
   The constant $f$ is chosen so that $e^{-f}$ was equal to the machine
   precision we work with (in practice then, $e^{-f}\sim 10^{-16}$).
%=============================================================================
\section{\label{sec:properties_spherical_harmonics}
   Properties of spherical harmonics
   }
As we are working with scalar fields on a topologically spherical set
of coordinates, we find it convenient to compute angular derivatives
of the scalar fields using a pseudospectral expansion
in spherical harmonics
\begin{align}
   \psi
   =&
   \sum_{lm} \psi_{lm}Y_{lm}\left(\theta,\phi\right)
   ,
\end{align}
where
\begin{align}
   Y_{lm}\left(\theta,\phi\right)
   \equiv&
   \sqrt{
      \frac{2l+1}{4\pi}
   }
   \sqrt{
      \frac{(l-m)!}{(l+m)!}
   }
   P^m_l\left(\mathrm{cos}\theta\right)
   e^{im\phi}
   ,\\
   P^m_l\left(x\right)
   \equiv&
   \left(-1\right)^m
   \left(1-x^2\right)^{m/2}\frac{d^m}{dx^m}P_l\left(x\right)
   ,
\end{align}
notice we have used the Condon-Shortly phase convention in the
definition of the associated Legendre functions.
The most convenient thing about this expansion is that it can be
very accurate for a sparse amount of data, and all the singular
$1/\mathrm{sin}\theta$ terms in the derivative operators
dissapear in the spectral representation of the derivatives.
Morever, given the utility of spherical harmonic representations in a 
wide variety of fields, there are several very fast libraries to 
go to/from spherical harmonic space.
Our code uses the library \texttt{shtns} \cite{2013GGG....14..751S}.

The rest of this section serves as a review of some basic facts about
the two sphere $\mathbb{S}_2$ and the spherical harmonics $Y_{lm}$.
This is all textbook material;
I mostly follow the presentation of Appendix A of \cite{Baumann:2019eav}.
As above, we work in spherical polar coordinates
\begin{align}
   d\sigma^2
   =
   d\theta^2
   +
   \mathrm{sin}^2\theta d\phi^2
   .
\end{align}
We write the three Killing vectors on the sphere as
\begin{subequations}
\begin{align}
   k_x
   \equiv&
   - 
   i\left(y\partial_z-z\partial_y\right)
   =
   i\left(
      \mathrm{sin}\phi\partial_{\theta}
      +
      \mathrm{cot}\theta\mathrm{cos}\phi\partial_{\phi}
   \right)
   ,\\
   k_y
   \equiv&
   -
   i\left(z\partial_x-x\partial_z\right)
   =
   i\left(
      -
      \mathrm{cos}\phi\partial_{\theta}
      +
      \mathrm{cot}\theta\mathrm{sin}\phi\partial_{\phi}
   \right)
   ,\\
   k_z
   \equiv&
   -
   i\left(x\partial_y-y\partial_x\right)
   =
   -
   i\partial_{\phi}
   .
\end{align}
\end{subequations}
In addition to being Killing vectors of $\mathbb{S}_2$, they are generators
of the group $SO(3)$, i.e. they satisfy the algebra
\begin{align}
   \left[k_i,k_j\right]
   =
   i\epsilon_{ijk}k_k
   ,
\end{align}
where $i,j,k\in\{1,2,3\}$ label
$x\leftrightarrow1$,
$y\leftrightarrow2$,
$z\leftrightarrow3$.
The algebra has the Casimir operator
\begin{align}
   L^2
   \equiv 
   L_x^2+L_y^2+L_z^2
   =
   -
   \Delta_{\mathbb{S}_2}
   ,
\end{align}
where $L_i$ is the Lie derivative along $k_i$, and
$\Delta_{\mathbb{S}_2}$ is the spherical Laplace-Beltrami operator.
The spherical harmonics form an irreducible representation of $SO(3)$,
and satisfy
\begin{subequations}
\begin{align}
   L^2Y^m_l
   =&
   \left(l+1\right)l
   ,\\
   L_zY^m_l
   =&
   mY^m_l
   .
\end{align}
\end{subequations}
The raising/lowering operators of the algebra $SO(3)$ are
\begin{align}
   L_{\pm}
   \equiv&
   k_x
   \pm
   i k_y
   \nonumber\\
   =&
   e^{\pm i\phi}
   \left(
      \pm
      \frac{\partial}{\partial\theta}
      +
      i\mathrm{cot}\theta
      \frac{\partial}{\partial\phi}
   \right)
   .
\end{align}
With the Condon-Shortley phase convention for the spherical harmonics, we have
\begin{align}
   L_{\pm}Y_{lm}
   =&
   \sqrt{(l\mp m)(l\pm m+1)}Y_{l (m\pm1)}
   .
\end{align}
With these relations, we can straightforwardly evaluate
the following derivatives in spectral space
\begin{subequations}
\begin{align}
   \frac{1}{\mathrm{sin}\theta}
   \frac{\partial}{\partial\theta}
   \left(
      \mathrm{sin}\theta\frac{\partial\psi}{\partial\theta}
   \right)
   +
   \frac{1}{\mathrm{sin}^2\theta}
   \frac{\partial^2f}{\partial\phi^2}
   =&
   -
   L^2\psi
   ,\\
   \left(\frac{\partial\psi}{\partial\theta}\right)^2
   +
   \left(
      \frac{1}{\mathrm{sin}\theta}
      \frac{\partial\psi}{\partial\phi}
   \right)^2
   =&
   -
   \left(L_+\psi\right)\left(L_-\psi\right)
   +
   \left(\partial_{\phi}\psi\right)^2
   .
\end{align}
\end{subequations}
%=============================================================================
\bibliography{thebib.bib}
%=============================================================================
\end{document}
